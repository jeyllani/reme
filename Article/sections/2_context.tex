\section{Contexte}
\label{sec:context}
Les politiques de taxation du carbone ont gagné en popularité en tant qu'outils essentiels pour lutter contre le changement climatique. Plusieurs études ont examiné l'efficacité de ces taxes dans divers contextes géographiques et économiques, fournissant des preuves empiriques de leurs impacts sur les émissions de gaz à effet de serre.

Notre étude se distingue par son approche rigoureuse de l'évaluation de l'impact de la taxe CO2 introduite en 2008 en Suisse. Bien que nous n'ayons pas trouvé d'effet causal clair, nous avons utilisé des méthodes économétriques robustes pour analyser et interpréter les données de manière transparente. Cette approche nous a permis de mieux comprendre les dynamiques complexes entourant cette politique et de fournir des insights précieux pour les décideurs politiques et les chercheurs intéressés par les mécanismes de tarification du carbone. En reconnaissant et en abordant les limitations méthodologiques, notre étude contribue de manière significative à la littérature existante et propose des pistes d'amélioration pour les futures recherches dans ce domaine.


\subsection{Littérature}
\label{subsec:litterature}

La taxation du carbone est largement reconnue comme un instrument politique efficace pour réduire les émissions de gaz à effet de serre. De nombreuses études empiriques ont examiné l'impact de ces taxes dans divers contextes, démontrant des effets variés en fonction des spécificités géographiques et économiques. Par exemple, une étude sur la Colombie-Britannique a montré que la mise en place d'une taxe carbone a entraîné une réduction des émissions de transport, bien que les réductions totales ne soient pas encore statistiquement significatives (Felix Pretis, 2022) \supercite{felix2022}. En Europe, une analyse macroéconomique des taxes carbone a révélé une réduction cumulative des émissions de 4 à 6 \% pour une taxe de 40 \$ par tonne de CO2, sans impact négatif significatif sur la croissance du PIB ou l'emploi (Metcalf, Gilbert E. and Stock, James H., 2020) \supercite{macro_impact}.

Les études ont également mis en lumière l'efficacité des réformes fiscales environnementales. Par exemple, une analyse utilisant la méthode de correspondance des scores de propension a démontré que l'adoption de la taxe carbone en Europe stimule significativement la réduction des émissions de CO2, soulignant le rôle crucial des politiques fiscales dans l'amélioration de la qualité environnementale (Ghazouani, Xia, Ben Jebli, \& Shahzad, 2020) \supercite{Ghazouani2020}. En Australie, l'introduction d'une taxe carbone a conduit à une réduction des émissions de gaz à effet de serre de 1,4 \% dans la deuxième année, malgré une augmentation des coûts de l'électricité (Centre for Public Impact, 2012) \supercite{CentrePublicImpact2012}. Aux États-Unis, une étude publiée dans \textit{Climate Change Economics} a estimé qu'une taxe de 50 \$ par tonne de CO2 pourrait réduire les émissions de gaz à effet de serre de 63 \% d'ici 2050, à condition que la taxe augmente de 5 \% par an (Fawcett, Mcfarland, Morris, & Weyant, 2018)\supercite{Fawcett2018}. Finkelstein-Shapiro et Metcalf (2022) \supercite{Finkelstein} ont exploré les effets macroéconomiques de cette politique, démontrant qu'elle peut être conçue pour réduire les émissions sans nuire à la croissance économique.

D'autres études ont exploré les réponses comportementales aux taxes carbone. Par exemple, une étude publiée dans SSRN a révélé des changements notables dans les actions ou les habitudes des individus en réponse à l'introduction des taxes carbone (Grieder, Baerenbold, Schmitz, \& Schubert, 2021)\supercite{grieder2021}. Le Journal of Environmental Economics and Management a également publié des recherches montrant l'efficacité des taxes sur le carbone dans divers contextes économiques et géographiques. Par exemple, une étude de Caron, Cohen, Brown, et Reilly (2018) a exploré les impacts d'une taxe nationale sur le CO2 aux États-Unis et les options de réinvestissement des revenus. Cette étude a révélé que des réductions significatives des émissions de gaz à effet de serre peuvent être atteintes avec des coûts relativement faibles, tout en tenant compte des impacts sur le bien-être des ménages et la distribution des revenus (Caron et al., 2018).\supercite{caron} En ce qui concerne les pays nordiques,  une étude de Bruvoll et Larsen (2002) a examiné les émissions de gaz à effet de serre en Norvège et a révélé que les taxes sur le carbone, bien que modestes dans leur effet direct, ont contribué à une réduction de 2\% des émissions de CO2, tandis que d'autres facteurs tels que la réduction de l'intensité énergétique et les changements dans le mix énergétique ont eu un impact plus significatif (Bruvoll \& Larsen, 2002).\supercite{bruvoll}.  Enfin, l'article intitulé "Pricing Carbon and the Social Cost of Carbon" par Martin Vrolijk et Misato Sato dans The World Bank Research Observer (2023) examine la relation entre la tarification du carbone et les émissions. L'étude trouve une corrélation significative entre les prix du carbone et la réduction des émissions, renforçant l'efficacité de la tarification du carbone comme outil de politique climatique (Vrolijk \& Sato, 2023) \supercite{vrolijk2023}.

Notre étude s'inscrit dans cette riche littérature en apportant une évaluation rigoureuse de l'impact de la taxe CO2 introduite en 2008 en Suisse, un pays souvent considéré comme un pionnier dans les politiques climatiques. Bien que nous n'ayons pas trouvé d'effet causal clair sur la réduction des émissions de gaz à effet de serre, nos analyses ont révélé des défis méthodologiques significatifs, tels que la violation de l’hypothèse des tendances parallèles. Ces résultats soulignent la complexité de l'évaluation des politiques climatiques et la nécessité de méthodes encore plus robustes pour isoler les effets causals. En réponse à ces défis, nous proposons plusieurs améliorations méthodologiques, notamment l'utilisation de modèles alternatifs et de données supplémentaires, pour renforcer la validité des conclusions futures. Nos résultats fournissent également des recommandations pratiques pour les décideurs politiques et les chercheurs. Ils mettent en évidence l'importance d'une évaluation rigoureuse et transparente des instruments de tarification du carbone et encouragent la poursuite de recherches approfondies pour mieux comprendre les dynamiques sous-jacentes des politiques climatiques. Bien que cette étude n'ait pas pu établir une relation causale claire, elle contribue de manière significative à la littérature existante en identifiant les défis méthodologiques et en proposant des solutions pour les surmonter. Ces insights sont précieux pour l'élaboration de futures politiques climatiques et pour l'amélioration continue des méthodes de recherche dans ce domaine.



