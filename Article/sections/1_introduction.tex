\section{Introduction}
\label{sec:introduction}

Le changement climatique représente un défi majeur et complexe de notre époque, avec des implications profondes pour les écosystèmes, la biodiversité et les sociétés humaines. Les activités humaines, principalement la combustion de combustibles fossiles et la déforestation, ont entraîné une augmentation des concentrations de gaz à effet de serre, provoquant une hausse des températures mondiales (IPCC, 2021)\supercite{ipcc_2021}. Ce réchauffement a des conséquences significatives, exacerbant les conditions météorologiques extrêmes et menaçant les moyens de subsistance dans le monde entier (Masson-Delmotte et al., 2021)\supercite{ipcc_2021}.



Récemment, la Suisse a été condamnée par la Cour européenne des droits de l'homme (CEDH) pour son inaction climatique, un verdict historique qui souligne l'importance de la responsabilité des états dans la lutte contre le changement climatique (CEDH, 2024)\supercite{CEDH2024}. Cette décision fait suite à une plainte déposée par le groupe KlimaSeniorinnen, qui a soutenu que le manque d'action climatique mettait en danger la vie des citoyens âgés, particulièrement vulnérables aux vagues de chaleur (Nature, 2024)\supercite{nature2024}.

Cependant, il est crucial de reconnaître les efforts significatifs de la Suisse en matière de politiques climatiques. Depuis l'introduction de la taxe sur le CO2 en 2008, la Suisse a mis en oeuvre plusieurs mesures pour réduire les émissions de gaz à effet de serre. La taxe sur le CO2, qui est l'une des plus élevées au monde, a été progressivement augmentée pour encourager la réduction de l'utilisation des combustibles fossiles et promouvoir des sources d'énergie plus durables (BAFU, 2024)\supercite{bafu2024}.


En 2021, la Suisse a adopté une révision majeure de sa Loi sur le CO2, visant à réduire encore davantage les émissions de CO2 d'ici 2030 et à atteindre la neutralité carbone d'ici 2050. Cette loi prévoit une combinaison d'incitations financières, d'investissements dans les technologies propres et de réglementations pour réduire les émissions dans tous les secteurs économiques (Conseil Fédéral, 2020; Conseil Fédéral, 2021; Swissinfo, 2021)\supercite{adminch2020,adminch2021,swissinfo2021}. De plus, depuis 2020, le système d'échange de quotas d'émission Suisse a été lié à celui de l'Union européenne, renforçant ainsi l'efficacité des mécanismes de réduction des émissions (Conseil Fédéral, 2020)\supercite{adminch2020}. Ces initiatives montrent que la Suisse n'est pas restée inactive face à la crise climatique. Au contraire, elle a pris des mesures ambitieuses et continues pour améliorer ses politiques environnementales.




Notre étude vise à évaluer l’impact de la taxe CO2 introduite en 2008 sur les émissions de gaz à effet de serre en Suisse, en utilisant des approches économétriques robustes. Nous avons entrepris cette recherche pour comprendre l’efficacité des politiques de tarification du carbone et pour fournir des preuves empiriques sur leur capacité à réduire les émissions de gaz à effet de serre. En analysant les données avant et après l’implémentation de la taxe, nous cherchons à déterminer si cette mesure a contribué à une diminution significative des émissions. L’objectif est de soutenir l’élaboration de politiques climatiques fondées sur des preuves, d’informer les décideurs et d’encourager d’autres pays à adopter des mesures similaires.


Nous formulons trois hypothèses principales pour guider notre recherche. Premièrement, nous supposons que l’introduction de la taxe CO2 en 2008 a entraîné une réduction significative des émissions de gaz à effet de serre en Suisse. Deuxièmement, nous hypothétisons que la taxe CO2 a conduit à des changements dans les comportements économiques, tels que la réduction de la consommation de combustibles fossiles et l’adoption de sources d’énergie plus durables. Enfin, nous anticipons que les effets de la taxe CO2 sont plus prononcés dans certains secteurs économiques, tels que le transport et l’industrie, par rapport à d’autres.

Nos résultats montrent une réduction des émissions de gaz à effet de serre, mais il est important de noter que la causalité précise entre la taxe et cette réduction n’a pas pu être établie de manière robuste en raison des défis méthodologiques, notamment la violation de l’hypothèse des tendances parallèles. Ces conclusions doivent donc être interprétées avec prudence. En dépit de ces défis, notre recherche offre des insights précieux et propose des recommandations pratiques pour les futures politiques climatiques, en soulignant l’importance d’une évaluation rigoureuse et transparente des instruments de tarification du carbone. Des recherches supplémentaires sont nécessaires pour confirmer ces conclusions et mieux comprendre les mécanismes sous-jacents (IMF, 2023)\supercite{imf2023}.

