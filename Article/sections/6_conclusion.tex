\section{Conclusion}
\label{sec:conclusion}

Cette étude examine l'impact de l'introduction de la taxe CO2 en 2008 en Suisse sur les émissions de gaz à effet de serre. En utilisant une approche de différence en différences (DiD) et la méthode de contrôle synthétique (SCM), nous avons constaté une réduction significative des émissions de CO2 par habitant en Suisse. Cependant, les défis méthodologiques, tels que la violation de l'hypothèse des tendances parallèles dans l'analyse DiD, limitent la robustesse et la possibilité d'émettre une conclusion causale.

Les pondérations allouées montrent que l'Autriche contribue à 90,2 \% dans le groupe de contrôle synthétique, tandis que le Luxembourg et la Hollande contribuent respectivement à 0 \% et 9,8 \%. Cette distribution des pondérations indique que l'Autriche est le principal pays de référence, reflétant des similitudes significatives entre les variables clés de l'Autriche et de la Suisse. Cependant, la faible contribution de la Hollande et l'absence de contribution du Luxembourg suggèrent que ces pays n'apportent pas suffisamment d'informations supplémentaires, limitant ainsi la robustesse du groupe de contrôle synthétique. En raison de la forte pondération de l'Autriche et des limitations méthodologiques observées, nous nous trouvons dans l’impossibilité de conclure à un effet causal robuste de la taxe CO2 sur les émissions de gaz à effet de serre en Suisse.

Notre analyse présente plusieurs limitations. Premièrement, la violation de l'hypothèse des tendances parallèles dans l'analyse DiD remet en question la robustesse des conclusions causales. Deuxièmement, la méthode de contrôle synthétique (SCM) présente des limitations, notamment en raison de la forte pondération de l'Autriche dans le groupe de contrôle, ce qui peut affecter la robustesse des conclusions. Troisièmement, la pandémie de COVID-19 a perturbé les émissions de CO2 après 2018, nous obligeant à exclure les données post-2018, ce qui pourrait influencer la généralisation des résultats. De plus, certaines variables potentiellement influentes, telles que la consommation énergétique, ont été exclues de l'analyse en raison de données manquantes.



Malgré ces limitations, notre recherche offre des insights précieux et propose des recommandations pratiques pour les futures politiques climatiques. Les résultats suggèrent que la taxe sur le carbone peut avoir un effet réducteur sur les émissions de CO2, mais des recherches supplémentaires sont nécessaires pour confirmer ces conclusions et mieux comprendre les mécanismes sous-jacents.

Bien que la réduction observée des émissions de CO2 soit encourageante, l'établissement d'une relation causale précise reste complexe en raison des limitations méthodologiques. Les décideurs doivent prendre en compte les défis liés à la mise en œuvre et à l'évaluation des politiques de tarification du carbone. Cela implique de concevoir des politiques qui non seulement encouragent la réduction des émissions, mais aussi minimisent les impacts économiques négatifs. Il est également crucial de s'assurer que les politiques sont basées sur des données de haute qualité et des analyses rigoureuses pour maximiser leur efficacité et leur acceptabilité. En particulier, les décideurs politiques doivent interpréter ces résultats avec prudence, tout en continuant à améliorer les méthodologies d'évaluation des politiques climatiques. En renforçant les méthodologies d'évaluation et en utilisant des données plus complètes, les politiques de tarification du carbone pourront être optimisées pour mieux répondre aux défis environnementaux et économiques.

En somme, cette étude souligne l'importance d'une évaluation rigoureuse et transparente des instruments de tarification du carbone, et appelle à des efforts continus pour affiner les méthodologies et améliorer la qualité des données utilisées. Cela permettra de soutenir l'élaboration de politiques climatiques fondées sur des preuves solides et d'encourager d'autres pays à adopter des mesures similaires pour lutter contre le changement climatique.




% \section{implication}
% Les implications de cette étude pour les décideurs politiques sont cruciales. Bien que nos résultats suggèrent que la taxe sur le carbone puisse effectivement réduire les émissions de CO2, il est essentiel de considérer ces résultats avec prudence en raison des limitations méthodologiques. La forte pondération de l'Autriche dans le groupe de contrôle synthétique et la contribution négligeable du Luxembourg et de la Hollande montrent qu'il est nécessaire d'améliorer la sélection et la composition des groupes de contrôle pour obtenir des évaluations plus robustes et fiables. En outre, ces résultats soulignent l'importance de renforcer les méthodologies d'évaluation pour mieux isoler les effets causals des politiques climatiques.

% En particulier, les décideurs doivent prendre en compte les défis liés à la mise en œuvre et à l'évaluation des politiques de tarification du carbone. Cela implique de concevoir des politiques qui non seulement encouragent la réduction des émissions, mais aussi minimisent les impacts économiques négatifs. Il est également crucial de s'assurer que les politiques sont basées sur des données de haute qualité et des analyses rigoureuses pour maximiser leur efficacité et leur acceptabilité.



\subsection{Suggestions pour les recherches futures}
\label{subsec:suggestion}

 
Pour les recherches futures, il est recommandé d'inclure plus de données en considérant les impacts dus à la pandémie de COVID-19, tout en gardant à l'esprit que les données peuvent être biaisées par les perturbations économiques et sociales liées à la pandémie. L'incorporation de variables supplémentaires, telles que des indicateurs de politiques environnementales spécifiques et des mesures de l’efficacité énergétique, pourrait également enrichir les analyses. De plus, l'application de méthodes économétriques alternatives, telles que les modèles à effets fixes avec variables instrumentales ou les méthodes de discontinuité de la régression (RD), pourrait fournir des perspectives complémentaires et aider à surmonter les défis méthodologiques actuels.

Ces améliorations méthodologiques permettront de soutenir l'élaboration de politiques climatiques fondées sur des preuves solides et encourageront d'autres pays à adopter des mesures similaires pour lutter contre le changement climatique. En continuant à affiner les méthodes de recherche et en améliorant la qualité des données utilisées, les futures études pourront offrir des insights plus précis et aider à développer des stratégies efficaces pour la réduction des émissions de gaz à effet de serre à l'échelle mondiale.