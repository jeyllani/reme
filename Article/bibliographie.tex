\begin{filecontents}{\references.bib}
@book{ipcc_2021,
  author    = {Masson-Delmotte, V. and Zhai, P. and Pirani, A. and Connors, S.L. and Péan, C. and Berger, S. and Caud, N. and Chen, Y. and Goldfarb, L. and Gomis, M.I. and Huang, M. and Leitzell, K. and Lonnoy, E. and Matthews, J.B.R. and Maycock, T.K. and Waterfield, T. and Yelekçi, O. and Yu, R. and Zhou, B. (eds.)},
  title     = {Climate Change 2021: The Physical Science Basis. Contribution of Working Group I to the Sixth Assessment Report of the Intergovernmental Panel on Climate Change},
  year      = {2021},
  publisher = {Cambridge University Press},
  address   = {Cambridge, United Kingdom and New York, NY, USA},
  pages     = {2391},
  doi       = {10.1017/9781009157896},
  note      = {IPCC}
}

@misc{CEDH2024,
  author       = {{Cour européenne des droits de l'homme}},
  title        = {Verein KlimaSeniorinnen Schweiz and Others v. Switzerland},
  year         = 2024,
  howpublished = {http://tiny.cc/i2tcyz},
  note         = {Affaire no 53600/20, Jugement du 9 avril 2024}
}



@article{nature2024,
  author = {Charlotte E. Blattner},
  title = {European ruling linking climate change to human rights could be a game changer — here’s how},
  journal = {Nature},
  year = {2024},
  volume = {628},
  pages = {691},
  doi = {10.1038/d41586-024-01177-3},
  url = {https://www.nature.com/articles/d41586-024-01177-3}
}


@article{bafu2024,
  author = {},
  title = {CO2 levy},
  journal = {Federal Office for the Environment},
  year = {2024},
  url = {https://www.bafu.admin.ch/bafu/en/home/topics/climate/info-specialists/co2-levy.html}
}

@article{swissinfo2021,
  author = {},
  title = {Swiss approve net-zero climate law},
  journal = {SWI swissinfo.ch},
  year = {2021},
  url = {https://www.swissinfo.ch/eng/swiss-approve-net-zero-climate-law/46711910}
}


@misc{adminch2020,
  author       = {{The Federal Council}},
  title        = {Linking of the Swiss and EU Emissions Trading Systems},
  year         = 2020,
  howpublished = {\url{https://www.admin.ch/content/gov/en/start/documentation/media-releases.msg-id-77037.html}},
  note         = {Accessed: 2024-05-27}
}


@misc{adminch2021,
  author       = {{The Federal Council}},
  title        = {CO2 Act and Long-Term Climate Strategy},
  year         = 2021,
  howpublished = {\url{https://www.admin.ch/gov/en/start/documentation/media-releases/media-releases-federal-council.msg-id-82140.html}},
  note         = {Accessed: 2024-05-27}
}


@article{euractiv2024,
  author = {},
  title = {Top Europe court chides Switzerland in landmark climate ruling},
  journal = {Euractiv},
  year = {2024},
  url = {https://www.euractiv.com/section/climate-environment/news/top-europe-court-chides-switzerland-in-landmark-climate-ruling/}
}

@article{abadie2010,
  author = {Abadie, A., Diamond, A., Hainmueller, J.},
  title = {Synthetic Control Methods for Comparative Case Studies: Estimating the Effect of California’s Tobacco Control Program},
  journal = {Journal of the American Statistical Association},
  year = {2010},
  volume = {105},
  number = {490},
  pages = {493-505},
  doi = {10.1198/jasa.2009.ap08746},
  url = {https://www.tandfonline.com/doi/abs/10.1198/jasa.2009.ap08746}
}

@article{imf2023,
  author = {International Monetary Fund. European Dept.},
  title = {Switzerland: Climate Change Mitigation in Switzerland},
  journal = {IMF Staff Country Reports},
  year = {2023},
  volume = {2023},
  number = {197},
  doi = {10.5089/9798400243608.002},
  url = {https://www.elibrary.imf.org/view/journals/002/2023/197/article-A001-en.xml}
}

@article{springer2023,
  author = {Author},
  title = {Does a Carbon Tax Reduce CO2 Emissions? Evidence from British Columbia},
  journal = {Springer},
  year = {2023},
  url = {https://link.springer.com/article/10.1007/springer2023}
}


@article{felix2022,
  author={Felix Pretis},
  title={{Does a Carbon Tax Reduce CO2 Emissions? Evidence from British Columbia}},
  journal={Environmental \& Resource Economics},
  year=2022,
  volume={83},
  number={1},
  pages={115-144},
  month={09},
  keywords={Carbon tax; CO2 emissions; Regulation; Break detection},
  doi={10.1007/s10640-022-00679-},
  abstract={ Using difference-in-differences, synthetic control, and introducing a new break-detection approach, I show that the introduction of North America’s first major carbon tax has reduced transportation emissions but not ‘yet’ led to large statistically significant reductions in aggregate CO2 emissions. Proposing a new method to assess policy based on breaks in difference-in-differences using machine learning, I demonstrate that neither carbon pricing nor trading schemes in other provinces are detected as large and statistically significant interventions. Instead, closures and efficiency-improvements in emission-intense industries in untaxed provinces have reduced emissions. Overall, the results show that existing carbon taxes (and prices) are likely too low to be effective in the time frame since their introduction.},
  url={https://ideas.repec.org/a/kap/enreec/v83y2022i1d10.1007_s10640-022-00679-w.html}
}

@techreport{macro_impact,
 title = "The Macroeconomic Impact of Europe’s Carbon Taxes",
 author = "Metcalf, Gilbert E and Stock, James H",
 institution = "National Bureau of Economic Research",
 type = "Working Paper",
 series = "Working Paper Series",
 number = "27488",
 year = "2020",
 month = "07",
 doi = {10.3386/w27488},
 URL = "http://www.nber.org/papers/w27488",
 abstract = {Policy makers often express concern about the impact of carbon taxes on employment and GDP. Focusing on European countries that have implemented carbon taxes over the past 30 years, we estimate the macroeconomic impacts of these taxes on GDP and employment growth rates for various specifications and samples.  Our point estimates suggest a zero to modest positive impact on GDP and total employment growth rates.  More importantly, we find no robust evidence of a negative effect of the tax on employment or GDP growth.  We examine evidence on whether the positive effects might stem from countries that used the carbon tax revenues to reduce other taxes; while the evidence is consistent with this view, it is inconclusive. We also consider the impact of the taxes on emission reductions and find a cumulative reduction on the order of 4 to 6 percent for a \$40/ton CO2 tax covering 30\% of emissions.  We argue that reductions would likely be greater for a broad-based U.S. carbon tax since European carbon taxes do not include in the tax base those sectors with the lowest marginal costs of carbon pollution abatement.},
}


@article{Ghazouani2020,
  title={Exploring the Role of Carbon Taxation Policies on CO2 Emissions: Contextual Evidence from Tax Implementation and Non-Implementation European Countries},
  author={Assaad Ghazouani and Wanjun Xia and Mehdi Ben Jebli and Umer Shahzad},
  journal={Sustainability},
  volume={12},
  number={20},
  pages={8680},
  year={2020},
  doi={10.3390/su12208680},
  url={https://www.mdpi.com/2071-1050/12/20/8680}
}


@article{CentrePublicImpact2012,
  author = {Centre for Public Impact},
  title = {The Carbon Tax in Australia},
  year = {2012},
  url = 
  {https://www.centreforpublicimpact.org/case-study/carbon-tax-australia},
  note = {Accessed: 2024-05-27}
}

@article{Fawcett2018,
  author = {Allen A. Fawcett and James R. Mcfarland and Adele C. Morris and John P. Weyant},
  title = {Exploring The Impacts Of A National U.S. CO2 Tax And Revenue Recycling Options With A Coupled Electricity-Economy Model},
  journal = {Climate Change Economics},
  volume = {9},
  number = {01},
  pages = {1840015},
  year = {2018},
  doi = {10.1142/S2010007818400158},
  url = {https://www.worldscientific.com/doi/abs/10.1142/S2010007818400158}
}


@techreport{Finkelstein,
 title = "The Macroeconomic Effects of a Carbon Tax to Meet the U.S. Paris Agreement Target: The Role of Firm Creation and Technology Adoption",
 author = "Finkelstein Shapiro, Alan and Metcalf, Gilbert E",
 institution = "National Bureau of Economic Research",
 type = "Working Paper",
 series = "Working Paper Series",
 number = "28795",
 year = "2021",
 month = "05",
 doi = {10.3386/w28795},
 URL = "http://www.nber.org/papers/w28795",
 abstract = {We analyze the quantitative labor market and aggregate effects of a carbon tax in a framework with pollution externalities and equilibrium unemployment. Our model incorporates endogenous labor force participation and two margins of adjustment influenced by carbon taxes: (1) firm creation and (2) green production-technology adoption.  A carbon-tax policy that reduces carbon emissions by 35 percent – roughly the emissions reductions that will be required under the Biden Administration's new commitment under the Paris Agreement – and transfers the tax revenue to households generates mild positive long-run effects on consumption and output; a marginal increase in the unemployment and labor force participation rates; and an expansion in the number and fraction of firms that use green technologies. In the short term, the adjustment to higher carbon taxes is accompanied by gradual gains in output and consumption and a negligible expansion in unemployment. Critically, abstracting from endogenous firm entry and green-technology adoption implies that the same policy has substantial adverse short- and long-term effects on labor income, consumption, and output. Our findings highlight the importance of these margins for a comprehensive assessment of the labor market and aggregate effects of carbon taxes.},
}


@article{grieder2021,
  title={The Behavioral Effects of Carbon Taxes – Experimental Evidence},
  author={Grieder, Manuel and Bärenbold, Rebekka and Schmitz, Jan and Schubert, Renate},
 journal={SSRN},
  year={2021},
  doi={http://dx.doi.org/10.2139/ssrn.3628516},
  url={https://ssrn.com/abstract=3628516 }
}





@article{caron,
    author = {Caron, Justin and Cohen, Stuart M. and Brown, Maxwell and Reilly, John M.},
    title = {Exploring the Impacts of a National U.S. CO2 Tax and Revenue Recycling Options with a Coupled Electricity-Economy Model},
    journal = {Climate Change Economics},
    volume = {09},
    number = {01},
    pages = {1840015},
    year = {2018},
    doi = {10.1142/S2010007818400158},
    URL = {https://doi.org/10.1142/S2010007818400158},
    eprint = {https://doi.org/10.1142/S2010007818400158},
    abstract = {This paper provides a comprehensive exploration of the impacts of economy-wide CO2 taxes in the U.S. simulated using a detailed electric sector model [the National Renewable Energy Laboratory’s Regional Energy Deployment System (ReEDS)] linked with a computable general equilibrium model of the U.S. economy [the Massachusetts Institute of Technology’s U.S. Regional Energy Policy (USREP) model]. We implement various tax trajectories and options for using the revenue collected by the tax and describe their impact on household welfare and its distribution across income levels. Overall, we find that our top-down/bottom-up models affects estimates of the distribution and cost of emission reductions as well as the amount of revenue collected, but that these are mostly insensitive to the way the revenue is recycled. We find that substantial abatement opportunities through fuel switching and renewable penetration in the electricity sector allow the economy to accommodate extensive emissions reductions at relatively low cost. While welfare impacts are largely determined by the choice of revenue recycling scheme, all tax levels and schemes provide net benefits when accounting for the avoided global climate change benefits of emission reductions. Recycling revenue through capital income tax rebates is more efficient than labor income tax rebates or uniform transfers to households. While capital tax rebates substantially reduce the overall costs of emission abatement, they profit high income households the most and are regressive. We more generally identify a clear trade-off between equity and efficiency across the various recycling options. However, we show through a set of hybrid recycling schemes that it is possible to limit inequalities in impacts, particularly those on the lowest income households, at relatively little incremental cost.}
}


@article{bruvoll,
  author = {Bruvoll, Annegrete and Larsen, Bodil},
  year = {2004},
  month = {02},
  pages = {493-505},
  title = {Greenhouse gas emissions in Norway: Do carbon taxes work?},
  volume = {32},
  journal = {Energy Policy},
  doi = {10.1016/S0301-4215(03)00151-4}
}


@article{vrolijk2023,
    author = {Vrolijk, Kasper and Sato, Misato},
    title = "{Quasi-Experimental Evidence on Carbon Pricing}",
    journal = {The World Bank Research Observer},
    volume = {38},
    number = {2},
    pages = {213-248},
    year = {2023},
    month = {03},
    abstract = "{A growing literature suggests that carbon emissions are most efficiently reduced by carbon pricing. The evidence base on the effectiveness of market-based mechanisms, however, faces three key limitations: studies often (a) predict, rather than evaluate effects, (b) show large difference in findings, and (c) cannot always infer causal relations. Quasi-experimental studies can address these challenges by using variation in policies over time, space, or entities. This paper systematically reviews this new literature, outlines the benefits and caveats of quasi-experimental methodologies, and verifies the reliability and value of quasi-experimental estimates. The overall evidence base documents a causal effect between carbon pricing and emission reductions, with ambiguous effects on economic outcomes, and there are important gaps and inconsistencies. This review underscores that estimates should be interpreted with care because of: (a) inappropriate choice of method, (b) incorrect implementation of empirical analysis (e.g., violate identifying assumptions), and (c) data limitations. More cross-learning across studies and use of novel empirical strategies is needed to improve the empirical evidence base going forward.}",
    issn = {0257-3032},
    doi = {10.1093/wbro/lkad001},
    url = {https://doi.org/10.1093/wbro/lkad001},
    eprint = {https://academic.oup.com/wbro/article-pdf/38/2/213/50739234/lkad001.pdf},
}

@misc{greenhouse,
  author = {Organisation for Economic Co-operation and Development (OECD)},
  title = {Environment Database - Greenhouse Gas Emissions},
  year = {2023},
  url = {http://www.oecd.org/env/},
  note = {Données collectées à partir des Inventaires Nationaux soumis en 2023 à la Convention-cadre des Nations Unies sur les changements climatiques (CCNUCC, tableaux CRF) et des réponses au questionnaire de l'OCDE sur l'état de l'environnement. Consulté le 27 mai 2024}
}


@misc{oecd2023,
  author = {Organisation for Economic Co-operation and Development (OECD)},
  title = {(OECD) Trends in the Transport Sector},
  year = {2023},
  url = {https://www.oecd.org/transport/data/oecd-transport-database.htm},
  note = {Données collectées auprès des ministères des Transports, des offices nationaux de statistique et d'autres institutions désignées comme sources officielles. Consulté le 27 mai 2024}
}

@misc{oecd_population,
  author = {Organisation for Economic Co-operation and Development (OECD)},
  title = {OECD Population Statistics},
  year = {2016},
  url = {https://stats.oecd.org/Index.aspx?DataSetCode=ITF_INLAND_INFR},
  note = {Données collectées auprès des offices nationaux de statistique, Eurostat et les Nations Unies. Consulté le 27 mai 2024}
}


@misc{oecd_transport,
  author = {Organisation for Economic Co-operation and Development (OECD)},
  title = {(OECD) Transport Performance Indicators},
  year = {2023},
  url = {https://stats.oecd.org/Index.aspx?DataSetCode=ITF_INLAND_INFR},
  note = {Données collectées à partir de multiples sources. Consulté le 27 mai 2024}
}

@misc{gdp,
  author = {Organisation for Economic Co-operation and Development (OECD)},
  title = {Growth in GDP per Capita, Productivity and ULC},
  year = {2022},
  url = {http://www.oecd.org/productivity/},
  note = {Estimations basées sur les Comptes Nationaux de l'OCDE, les Perspectives de l'Emploi de l'OCDE et des sources nationales. Consulté le 27 mai 2024}
}

@misc{energy,
  author = {Ritchie, Hannah and Roser, Max},
  title = {Energy},
  year = {2023},
  url = {https://ourworldindata.org/energy},
  note = {Consulté le 27 mai 2024}
}

@article{stata_didregress,
  title = {didregress postestimation: Postestimation tools for didregress and xtdidregress},
  organization = {StataCorp LLC},
  year = {2021},
  url = {https://www.stata.com/manuals/tedidregresspostestimation.pdf}
}

@manual{chatgpt2023,
  title={ChatGPT V.4: OpenAI’s Language Model},
  author={OpenAI},
  year={2023},
  note={Correction orthographique et syntaxique},
  url= {https://www.openai.com/research/chatgpt}
}

@misc{mize2023,
  author = {Trenton Mize},
  title = {CleanPlots: A Software for Creating Clean and Aesthetically Pleasing Plots in Stata},
  year = {2023},
  url = {https://www.trentonmize.com/software/cleanplots},
  note = {Consulté le 27 mai 2024}
}


@misc{synth2023,
  author = {Jann, Ben and Hainmueller, Jens and Abadie, Alberto and Diamond, Alexis},
  title = {Synth: Stata Module to Implement Synthetic Control Methods for Comparative Case Studies},
  year = {2023},
  url = {http://fmwww.bc.edu/RePEc/bocode/s/synth.html},
  note = {Consulté le 27 mai 2024}
}



\end{filecontents}

\addbibresource{\references.bib}

